% Mooers2020ShortcutsForFasterImageCreationInPyMOL.tex

\subsubsection*{Summary} 
PyMOL is a tool for generating images of biomolecular structures, offering extensive control over their appearance through numerous parameters. \index{PyMOL}
To enhance and simplify its use, 241 Python functions, termed “shortcuts,” were developed. \index{Python}
These shortcuts, organized into 25 functional groups, allow users to perform tasks such as creating new molecular representations, saving files with unique time-stamped names, and conducting web searches directly from PyMOL. \index{molecular representaton}

\subsubsection*{Context in Molecular Graphics and Human-Computer Interactions} 
To streamline and enhance user interactions with PyMOL, 241 Python functions, known as “shortcuts,” were developed.\index{shortcuts} 
These shortcuts, categorized into 25 functional groups, facilitate tasks such as creating innovative molecular representations, saving files with unique time-stamped names to prevent overwriting, and conducting web searches directly from PyMOL. 
The help function provides documentation and reusable PyMOL commands, significantly improving user efficiency by reducing the time spent searching for code fragments.\index{code reuse}  
This integration of shortcuts exemplifies the synergy between molecular graphics and user-friendly interfaces, optimizing the workflow for researchers and scientists.
This is an excellent of good \gls{hci} \index{human-computer interactions}.